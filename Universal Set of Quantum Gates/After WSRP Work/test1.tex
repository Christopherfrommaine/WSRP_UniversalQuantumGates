\documentclass[12pt]{article}
\usepackage{graphicx}
\usepackage{amsmath, amssymb, amsthm}
\usepackage[letterpaper, margin=1in]{geometry}

\usepackage{circuitikz}
\usepackage{quantikz}
\usepackage{braket}

\usepackage{setspace}
\onehalfspacing

\usepackage{fancyhdr}
\fancyhf{}
\rfoot{\thepage}
\pagestyle{fancy}

\title{On Minimal Universal Quantum Gates}
\author{Christopher Gilbert}
\date{November 2024}

\newcommand{\nand}{\overline{\land}}
\newcommand{\nor}{\overline{\lor}}
\newcommand{\xor}{\oplus}
\newcommand{\nxor}{\overline{\xor}}


\begin{document}

\maketitle
\newpage

\begin{abstract}
This paper investigates the construction of universal classical and quantum gates, and proposes two novel quantum sets which use the fewest types of gates for each of their constructions of universality. For a circuit with ancilla qubits, a zero-controlled dual-axis irrational rotation gate is universal within arbitrary prescision, and for a maximally entangled circuit with no ancillas—a construction more similar to classical universality—the set containing the controlled 90-degree Z-rotation gate and the irrational Y-rotation gate is universal to any prescision.
\end{abstract}
\newpage

\section{Introduction}

\subsection{Universal Gates}
A logic gate is an operation which acts on bits or qubits. For example, the classical AND gate recieves two bits and outputs a 1 if and only if both input bits were 1. Let a gate $G$ that takes in $n$ inputs and returns $m$ outputs be denoted
$$G : n \to m$$
In conventional function notation, this would be written
$$G : \{0, 1\}^n \to \{0, 1\}^m$$
A gate or set of gates is universal if some arrangement of the gates can replicate the behavior of every possible gate.

An example of such a $2 \to 1$ gate for classical universality is the NAND gate, denoted $\nand$, which outputs a 0 if and only if both of its input bits are 1. The universality of NAND can be proved either by exhaustively showing that every possible $2 \to 1$ gate can be constructed by NAND, or with a more formal proof which will be presented later on.

A gate set is said to be minimal if it is the smallest gate set possible to make. For example, $\{ \nand \}$ is a minimal universal set because any computation is impossible without at least one operation, and so a set with only a single element is the smallest possible set to be universal.

\subsection{Quantum Gates}
\subsubsection{Quantum Computing}
A quantum computer uses the properties of quantum mechanics in order to perform calculations. By creating a superposition of multiple possible inputs, all operations in a circuit are simultaneously done on each input. By canceling out the probabilities of inputs with negative results using quantum phase, an operation can find only the correct inputs with a quadratic improvement over classical computers. For other less generalizable quantum algorithms, notably Shor's algorithm, the speedup over classical computers can be up to exponential.

\subsubsection{Quantum State Vectors}
The state of a classical system may be represented as a sequence of bits, where each bit represents a wire in a circuit. By interpreting this sequence of bits as a binary integer, every classical state can be represented with a single number.

Quantum states, on the other hand, may be in a superposition of multiple states and so may be represented as a list of probability amplitudes corresponding to each classical state. This list is called the state vector and can accurately represent any possible quantum superposition. Thus a classical state with three wires can be in one of eight states, and a quantum state with three qubits (the quantum equivalent of classical bits) can be fully described by a unit vector in $\mathbb{C}^8$. The state vector must always have a length of 1, because the sum of the squared amplitudes has to be 1, just as the sum of a set of probabilities must always add up to 1.

A classical state with state number $n$ may also be represented as the $n$th unit vector, because it has a 100\% chance to be in the state $n$.

\subsubsection{Quantum Operations}
The state vector representation of a classical or quantum state is extremely helpful in formulating universality mathematically using linear algebra. For an $n$-wire circuit with $N=2^n$ states, a matrix
$$
(\vec{c_{1}} \: \dots \: \vec{c_{N}})
$$
will take a classical state vector in the state $i$ to the state vector $\vec{c_{i}}$. Thus, any operation can be directly described as a matrix of the output state vectors for each input.

In almost all cases, these matrices are square, because most quantum operations are in-place, meaning they change the state but do not add or delete wires from a quantum circuit. Some exceptions include state preperation matrices (which are used to mathematically describe extra ancilla wires) and discarding matrices (which remove a 0-state wire from a circuit).

In most cases, an operation is done only on a subset of the wires in a circuit. In order to apply an operation matrix, it first needs to be combined with identity matrices until it achieves the correct dimensions. For a circuit with $n$ wires, in order to operate $O$ on wires $a$ through $b$, the following matrix $M$ must be applied to the state vector of the entire circuit:
$$
M_{a, b} = I_{2^a} \otimes O \otimes I_{2^{n - b}}
$$
where $\otimes$ is the Kronecker product and $I_x$ is the $x \times x$ identity matrix.

Each matrix $M$ represents one way that $O$ can be applied to a circuit, so the problem of universality can be described in terms of a matrix decomposition of any arbitrary operation into matrices of the form of $M$.

The proofs of universality presented in this paper will rely on decomposing each operation in an already proven quantum gate set into operations from a smaller gate set.

\section{Classical Universality}
\subsection{Classical Circuits}
Classical circuits are generally presented as electrical circuit diagrams. For example, a demonstration of how an XOR gate can be made from NAND gates is given in Figure 1.

\begin{figure}[h]
    \centering
    `\begin{circuitikz}
        \draw
        % Center NAND
        (0,0) node[nand port] (nand12) {}
        (nand12.in 1) -- ++(-1.5, 0) node[anchor=east] {$p$}
        (nand12.in 2) -- ++(-1.5, 0) node[anchor=east] {$q$}

        ($(-2, 0)!(nand12.in 1)!(-2, 0)$) node[circle, fill=black, minimum size=4pt, inner sep=0pt] (pnode) {}
        
        ($(-2, 0)!(nand12.in 2)!(-2, 0)$) node[circle, fill=black, minimum size=4pt, inner sep=0pt] (qnode) {}

        (0, 1.5) node[nand port] (nand11) {}
        (pnode) |- (nand11.in 1)
        
        (0, -1.5) node[nand port] (nand13) {}
        (qnode) |- (nand13.in 2)

        ($(pnode)!(nand11.in 2)!(pnode)$) node[circle, fill=black, minimum size=4pt, inner sep=0pt] (node11) {}
        -- (nand11.in 2)

        ($(qnode)!(nand13.in 1)!(qnode)$) node[circle, fill=black, minimum size=4pt, inner sep=0pt] (node13) {}
        -- (nand13.in 1)
        
        ($(2, 0)!(nand11.out)!(2, 0)$) +(0, -0.25) node[nand port] (nand21) {}
        (nand11.out) -- (nand21.in 1)
        (nand13.out) -| ($(nand13.out)!0.5!(nand21.in 2)$)
                    |- (nand21.in 2)
        
        ($(4, 0)!(nand12.out)!(4, 0)$) +(0, 0.25) node[nand port] (nand31) {}
        (nand12.out) -- (nand31.in 2)
        (nand21.out) -| ($(nand21.out)!0.5!(nand31.in 1)$)
                    |- (nand31.in 1)

        (6, 0) node[nand port] (nand41) {}
        (nand31.out) -- (nand41.in 1)

        ($(nand31.out)!0.5!(nand41.in 1)$) node[circle, fill=black, minimum size=4pt, inner sep=0pt] {}
        |- (nand41.in 2)

        (nand41.out) -- ++(0.5, 0) node[anchor=west] {$o$}
        
        ;
    \end{circuitikz}
    \caption{XOR Circuit made from NAND Gates (Created by Student Researcher)}
    \label{fig:classical xor from nand circuit}
\end{figure}

This classical diagram has a number of properties which differ from most quantum circtuits:
\begin{itemize}
    \item Wires are allowed to cross and move
    \item Wires may be copied or duplicated
    \item Gates may change the number of wires
    \item No ancilla wires (in the 0 state) are provided
\end{itemize}
In order to rectify the first disparity, all quantum circuits used to prove universal quantum gate sets will also be allowed to use the SWAP gate, represented with a $\times$ symbol on each wire, which swaps the state of two qubits and so is equivalent to crossing wires in a classical circuit.

The second disparity presents a more significant problem, because the No Cloning Theorem asserts that quantum states cannot be perfectly copied. Instead of directly copying states, another approach is to initialize a circuit with a number of wires in each state. Thus any previously required copying can be acheived throgh duplication of the applied operations.

The third disparity can be prevented by changing the types of classical gates used. Instead of $2 \to 1$ gates like NAND, one can implement $2 \to 2$ gates which replicate the behavior of the original gate on the lower output wire, and then have some other dummy behavior on the upper output wire. Under this new construction, when given an input of $(p, q)$, the modified NAND gate returns $(0, p \nand q)$. This simple example is not entirely accurate as quantum gates must also be reversible for reasons of physics, but this approach can generally be applied.

The final disparity can be solved by just dissallowing any ancilla wires in quantum circuits, but under most formulations of quantum universality, it is assumed that as many ancilla bits are given as are required. This is not so much a mathematical problem as it is a problem with differing definitions of universality. Because of this, this paper will explore both the case where ancilla bits in the 0 state are allowed.

$\emph{Fig. \ref{fig:quantum xor from nand circuit}}$ shows the new implementation of an XOR gate after making the above changes. It only uses modified $2 \to 2$ NAND gates, and it also shows the current state at various positions along the circuit, logically simplified as much as possible.

\begin{figure}[h]
    \centering
    \scalebox{0.8}{
        \begin{quantikz}
            p \: & \qw & \gate[2]{\text{NAND}} & \: 0 \: & \qw & \qw & \: 0 \: & \qw & \qw & \: 0 \: & \qw & \qw & \: 0 \: \\
            q \: & \qw & \qw & \: p \nand q \: & \qw & \qw & \: p \nand q \: & \swap{3} & \qw & \: 0 \: & \qw & \qw & \: 0 \: \\
            p \: & \qw & \gate[2]{\text{NAND}} & \: 0 \: & \qw & \qw & \: 0 \: & \qw & \qw & \: 0 \: & \qw & \qw & \: 0 \: \\
            q \: & \swap{1} & \qw & \: \neg p \: & \swap{1} & \qw & \: 0 \: & \qw & \qw &  \: 0 \: & \qw & \qw & \: 0 \: \\
            p \: & \swap{-1} & \gate[2]{\text{NAND}} & \: 0 \: & \swap{-1} & \gate[2]{\text{NAND}} & \: 0 \: & \swap{-3} & \gate[2]{\text{NAND}} &  \: 0 \: & \qw & \qw & \: 0 \: \\
            q \: & \qw & \qw & \: \neg q \: & \qw & \qw & \: p \nor q \: & \qw & \qw &  \: p \nxor q \: & \swap{5} & \qw & \: 0 \: \\
            p \: & \qw & \gate[2]{\text{NAND}} & \: 0 \: & \qw & \qw & \: 0 \: & \qw & \qw & \: 0 \: & \qw & \qw & \: 0 \: \\
            q \: & \qw & \qw & \: p \nand q \: & \qw & \qw & \: p \nand q \: & \swap{3} & \qw & \: 0 \: & \qw & \qw & \: 0 \: \\
            p \: & \qw & \gate[2]{\text{NAND}} & \: 0 \: & \qw & \qw & \: 0 \: & \qw & \qw & \: 0 \: & \qw & \qw & \: 0 \: \\
            q \: & \swap{1} & \qw & \: \neg p \: & \swap{1} & \qw & \: 0 \: & \qw & \qw &  \: 0 \: & \qw & \qw & \: 0 \: \\
            p \: & \swap{-1} & \gate[2]{\text{NAND}} & \: 0 \: & \swap{-1} & \gate[2]{\text{NAND}} & \: 0 \: & \swap{-3} & \gate[2]{\text{NAND}} &  \: 0 \: & \swap{-5} & \gate[2]{\text{NAND}} & \: 0 \: \\
            q \: & \qw & \qw & \: \neg q \: & \qw & \qw & \: p \nor q \: & \qw & \qw &  \: p \nxor q \: & \qw & \qw & \: p \xor q \: \\
        \end{quantikz}}
    \caption{Quantum-Compatible XOR from NAND (Source: Christopher Gilbert)}
    \label{fig:quantum xor from nand circuit}
\end{figure}



\subsection{Proof of Classical Universality}
The most basic classical gates are the $1 \to 1$ gates: $I$ (Identity),  $\neg$ (NOT), $1$ (Constant 1), and $0$ (Constant 0). However, as none of these gates can relate one bit to another, none can perform all binary operations (Such as AND), so it is impossible for any to be universal. Because of this, a universal gate set must have at least one gate with at least two inputs.
With this limitation, the simplest possible gate type that could be universal is $2 \to 1$.

Focusing back to the regular classical case, without the modifications for implementing it in quantum circutry, this paper will now present a proof that NAND and NOR are the only $2 \to 1$ universal classical quantum gates. The intuition gained from this proof and the two conditions for universality that it sets forth will be useful in the following sections of the paper.


\newtheorem{theorem}{Theorem}
\newtheorem{lemma}{Lemma}
\newtheorem{corrolary}{Corrolary}

\begin{theorem}
    The only $2 \to 1$ universal classical gates are NAND and NOR.
\end{theorem}

\begin{lemma}
    For all universal $2 \to 1$ gates G, $G(x, x) = \neg x$ for all $x$.
\end{lemma}

\begin{proof}
    All universal $2 \to 1$ gates $G$ must be able to create any other possible gate, including the gate $H(x, y) = 1$.

    $H(0, 0) = 1$, so when given an input of $(0, 0)$, $G$ must be able to replicate this output in some way. If $G(0, 0) \ne 1$, then no matter how many times it is applied or the inputs are copied, all wires will be in the $0$ state. There would therefore never be any way to get an output in the $1$ state. By contradiction, $G(0, 0)$ must equal $1$.
    
    By the same logic, when given an input of $(1, 1)$, in order to have any wire in the $0$ state, $G(1, 1) = 0$.
    Therefore, because $G(0, 0) = 1$ and $G(1, 1) = 0$, $G(x, x) = \neg x$. 
\end{proof}

\begin{corrolary}
    Any $2 \to 1$ universal gate can replicate the behavior of the $1 \to 1$ not gate, as any input $x$ can be copied to $(x, x)$, onto which $G$ can be applied, resulting in $\neg x$.
\end{corrolary}

\begin{lemma}
    For all universal $2 \to 1$ gates $G$,
    $G(x, \neg x) = G(\neg x, x)$
\end{lemma}

\begin{proof}
	In order for a gate to be universal, it must be able to correlate two inputs together. That is, a gate $G$ which can be written as $G(p, q) = H(p)$ or $G(p, q) = H(q)$, for some other $1 \to 1$ gate $H$, cannot be universal. If $G$ can be represented in such a way, then the condition holds that it is invariant under at least one input. Therefore, any gate $G$ such that for all inputs p and q, either $G(p, q) = G(\neg p, q)$ or $G(p, q) = G(p, \neg q)$ is not universal. That is,

    $$(\forall (p, q), G(p, q) = G(\neg p, q) \lor G(p, q) = G(p, \neg q)) \implies G \text{ is not universal.}$$

    By proof of the contrapositive, this statement implies that if G is universal, then the following holds:
    \begin{align}
        &\neg (\forall (p, q), G(p, q) = G(\neg p, q) \lor G(p, q) = G(p, \neg q)) \\
        &\exists (p, q), \neg (G(p, q) = G(\neg p, q) \lor G(p, q) = G(p, \neg q)) \\
        &\exists (p, q), G(p, q) \ne G(\neg p, q) \land G(p, q) \ne G(p, \neg q)) \\
        &\exists (p, q), G(p, \neg q) \ne G(p, q) \ne G(\neg p, q) \\
        &\exists (p, q), G(p, \neg q) = G(\neg p, q)
    \end{align}

    For the sake of contradiction, when $p = \neg q$, (and thus $q = \neg p$),
    $$\exists p, G(p, p) = G(\neg p, \neg p)$$

    which, by Lemma 1, must be false for all universal gates. Therefore, if G is universal, it must satisfy the condition that:

    $$G(p, \neg p) = G(\neg p, p)$$
\end{proof}

\begin{proof}
    Of the 16 possible $2 \to 1$ gates, only NAND and NOR satisfy both the condition that $G(p, p) = \neg p$ and $G(p, \neg p) = G(\neg p, p)$.
    Both of these can be exhaustively proved to be universal for all $1 \to 1$ and $2 \to 1$ gates. E.g.
    \begin{align}
        \neg p &= p \nand p \\
        0(p) &= (p \nand p) \nand p \\
        p \land q &= (p \nand q) \nand (p \nand q) \\
        p \lor q &= (p \nand p) \nand (q \nand q) \\
        &\text{etc.}
    \end{align}
    In order to prove that NAND and NOR are fully universal (rather than universal for only $2 \to 2$ gates), one can show that it is possible to build a multiplexer to split every possible input state using only NOT (implied to be constructable by the Corrolary to Lemma 1), AND (proved above), and OR (proved above). For every possible inputs, one can select the desired output by either sending the wire $w$ into a $0(w)$ gate for an output of 0, or by sending the wire into an identity gate for an output of 1. Finally, using the OR gate, one can combine all the multiplexed wires back together into a single output.
    This algorithm is able to create any gate with an arbitrary number of inputs, and multiple $n \to 1$ gates may be put in paralell in order to create an $n \to m$ gate, just as any $f: \mathbb{R}^n \to \mathbb{R}^m$ function may be decomposed as $f(\overrightarrow{x}) = (f_1(\overrightarrow{x}), f_2(\overrightarrow{x}), \dots, f_m(\overrightarrow{x}))$.
    Using this approach, any arbitrary gate may be created only from NAND, and thus NAND is universal. The same logic may be applied to NOR to prove it's universality. Due to lemmas 1 and 2, these two are the only gates to satisfy both conditions, and so NAND and NOR are the only $2 \to 1$ universal classical gates.
\end{proof}

\section{Quantum Universality}
Just as all classical gates needed to be able to change single bits (i.e. implement the NOT gate, as was proven in the Corrolary to Lemma 1) as well as correlate bits together (as proved in Lemma 2), for a set of quantum gates to be universal, it must be able to make every possible change to an individual qubit's state as well as have some method to correlate or entangle multiple qubits together.

\subsection{Bloch Sphere}
Recall that a quantum state is represented by a statevector: a linear combination of the possible classical states of the system. This statevector is complex, due to quantum physics, and has a magnitude of 1, due to the summation of probabilities. Thus, the set of all possible quantum states is isomorphic to a sphere in $\mathbb{R}^4$. However, the complex phase of the  $\ket{00}$ (i.e.$\left[\begin{smallmatrix} 1 \\ 0 \\ 0 \\ 0 \end{smallmatrix}\right]$) state cannot be physically measured or determined, so it is factored out as a coefficient to the state vector which can then be discarded. This means that a quantum state is generally visualized as a vector on the unit sphere of $\mathbb{R}^3$, with single-qubit operations corresponding to rotations about the sphere.

\subsection{Minimal Universal Quantum Gates}
A trivial set of universal quantum gates is the set of all possible rotations in each axis plus a the entanglement gate CNOT:
$$\{R_X(a) : a \in \mathbb{R}\} \cup \{R_Y(b) : b \in \mathbb{R}\} \cup \{R_Z(c) : c \in \mathbb{R}\} \cup \{\text{CNOT}\}$$
This satisfies the above conditions for universality: it can construct any possible rotation for a single qubit and it can correlate two qubits' states together. Although this set can exactly implement any other possible quantum gate, it contains every possible axis rotation gate, which is far more gates than is necesary.

\subsubsection{Irrational Angle Gates}
A much smaller gate set can be constructed after noting the fact that repeated rotations by an irrational angle can approximate a rotation by any angle to within some arbitrary prescision efficiently. This fact can be derived from the Kronecker approximation theorem, and a more general result which applies to all possible gates is given by the Solvay-Kitaev theorem.
Such irrational angled rotations can be implemented with a sequence of rational rotations in different axes. For example, a rotation by $\pi/4$ in the Z axis then a roation by $\pi / 4$ in the X axis results in a rotation angle of
$$\arccos\left(\frac{\sqrt{2}}{2} - \frac{1}{4}\right) \approx 1.09606 \text{ about the axis } \langle 1, 1 - \sqrt{2}, 1 \rangle \text{.}$$
For example, the Clifford + T set, consisting of the gates $\{\text{CNOT}, \text{H}, \text{S}, \text{T}\}$, is universal despite the fact that the rotation angle of each single-qubit gate is rational.

\subsubsection{Reducing the Number of Gates}
In the proof of classical universality, two basic properties for universality were established: that any universal set must be able to perform a bitflip on any input, and that it must be able to correlate two wires together. Based on these two properties, one can construct a universal set of only two gates, each of which performs one of these actions.

The Pauli-Y gate can be used to perform a bitflip, and the CNOT gate can entangle two qubits, but although this satisfies the classical requirements, a quantum universal set must also be able to apply a phase-shift to the input. The simplest universal quantum gate set must then involve a bitflip gate (i.e. the Y gate) and a controlled arbitrary Z gate, as rotations about the Z-axis impart a phase on a state. Specifically with the 90-degree Y rotation gate (denoted either $R_Y(\pi/2)$ or $Y^{1/2}$), any rotation around the Z-axis may be mapped to a rotation on the X-axis. Rotations about the X and Z axes correspond the spherical coordinate system for a constant radius, which implies that these two gate types can indeed replicate any single-qubit rotation. This is the intuition behind the circuits that I will use to prove the universality of these gates.

Before going further, some clarification regarding the Z-axis rotation is requried. An arbitrary rotation gate must be able to apply an infinite number of different possible rotations, and thus cannot be considered a single gate itself. Instedad, one can use an irrational-angled gate to approximate any rotation to arbitrary prescision.

For example, an irrational Z rotation gate $R_Z(2\pi\xi)$ where $\xi$ is irrational can approximate any Z rotation gate. For my gate set, the golden ratio $\Phi$ will be used as the irrational number, as $n \Phi \mod 1$ is fairly evenly distributed across the unit interval.
This logic results in the 2 gate set:
$$\{CR_Z(2\pi\Phi), R_Y(\pi/2)\}$$
However, this is also where the two differing formulations of universality begin to give different results. While in the formulation with no ancilla, the controlled gate and the bitflip gate must be seperate gates, in the standard formulation allowing for ancillas, an even smaller gate set may be constructable. This difference will be considered later in the paper.

\subsubsection{Proof of Universality}
In order to prove that this set is universal, it must be shown that this gate set can implement any other gate. Because Clifford + T is universal, then if each gate of Clifford + T can be implemented, then it implies that the set is universal. That is, if a gate set can implement gates which can implement every circuit, then that gate set must be able to directly implement any circuit, and is therefore universal.

\paragraph{Proof of CNOT}
The matrix form of the CNOT gate is
$$
\begin{bmatrix}
    1&0&0&0\\
    0&1&0&0\\
    0&0&0&1\\
    0&0&1&0
\end{bmatrix}
$$
which can be implemented with the following circuit (where exponentiation represents repeated application):
\begin{figure}[h]
    \centering
    \scalebox{1.3}{
        \begin{quantikz}
            \ket{\psi_1} \: & \qw & \qw & \ctrl{1} & \qw & \qw \\
            \ket{\psi_2} \: & \qw & \gate{R_Y(\pi/2)^{3}} & \gate{R_Z(\pi)} & \gate{R_Y(\pi/2)^{1}} & \qw
        \end{quantikz}}
    \caption{Proof of CNOT Circuit}
    \label{fig:quantum universal set 1 proof of CNOT circuit algebraic}
\end{figure}

Multiplying the circuit matrices results in the CNOT matrix, except with a flipped global phase, which is inconsequential. As the arbitrary Z rotation gate has been replaced by the irrational Z rotation gate, $R_Z(\pi)$ must be approximated. With $\epsilon \le 10^{-5}$, the approximation comes out to be $R_Z(2\pi \Phi)^{98209}$. This specific exponent was found computationally through a brute force search. The net effect of this circuit is to apply this matrix:

$$
\scalebox{0.7}{$
\begin{bmatrix}
    0.9999999999999997&0.&0.&0.\\
    0.&0.9999999999999997&0.&0.\\
    0.&0.&0.00000000000957273-0.00000357644\mathbf{i}&0.9999999999904268+0.00000357644 \mathbf{i}\\
    0.&0.&0.9999999999904268 + 0.00000357644 \mathbf{i}&0.00000000000957273-0.00000357644\mathbf{i}
\end{bmatrix}
$}
$$
which approximates CNOT to $\epsilon = 7.15287 \times 10^{-6}$, within the desired accuracy.\footnote{The Solvay-Kitaev theorem is given in terms of the L2 matrix norm, but as the L2 norm bounds the Frobenius norm from below, the computationally simpler Frobenius norm will be used for epsilon values.}

\paragraph{Proof of T}
Implementing the T gate itself is fairly simple, because T is equivalent to $R_Z(\pi/2)$. However, $R_Z$ and $CR_Z$ are considered different gates, so in order to implement a $R_Z(\pi/2)$, a state in the $\ket{1}$ state must be prepared. Without the use of an ancilla qubit already in the $\ket{0}$ state, it can be done with the following circuit:
\begin{figure}[h]
    \centering
    \scalebox{1.1}{
        \begin{quantikz}
            \ket{\psi_{in}} \: & \qw & \ctrl{1} & \qw & \gate{R_Z(\pi/4)} & \qw & \: \ket{\psi_{out}}\\
            \ket{\psi_{in}} \: & \gate{R_Y(\pi/2)^3} & \gate{R_Z(\pi)} & \gate{R_Y(\pi/2)^3} & \ctrl{-1} & \gate{R_Y(\pi/2)^2} & \: \ket{0}
        \end{quantikz}}
    \caption{Proof of T Circuit, No $\ket{0}$ Ancilla}
    \label{fig:quantum universal set 1 proof of T circuit}
\end{figure}

If instead an ancilla is allowed, then it can be implemented simply with the following circuit, which prepares the ancilla from a $\ket{0}$ to a $\ket{1}$ state, applies the rotation, then reverts the ancilla:
\begin{figure}[h]
    \centering
    \scalebox{1.1}{
        \begin{quantikz}
            \ket{\psi_{in}} \: & \qw & \gate{R_Z(\pi/4)} & \qw & \: \ket{\psi_{out}}\\
            \ket{0} \: & \gate{R_Y(\pi/2)^2} & \ctrl{-1} & \gate{R_Y(\pi/2)^2} & \: \ket{0}
        \end{quantikz}}
    \caption{Proof of T Circuit, Ancilla Allowed}
    \label{fig:quantum universal set 1 proof of T circuit}
\end{figure}

Again, the $R_Z(\pi)$ gate is approximated with $R_Z(2\pi \Phi)^{98209}$, and $R_Z(\pi/4)$ is approximated with $R_Z(2\pi \Phi)^{5796}$. Excluding the ancilla bit, this circuit is to applies the matrix:

$$
\scalebox{1}{$
\begin{bmatrix}
    0.9999999999999993&0.\\
    0.&0.7071096086512368+0.7071003772806271\mathbf{i}
\end{bmatrix}
$}
$$
which approximates the T gate to $\epsilon = 7.00033 \times 10^{-6}$.


\paragraph{Proof of S}
The S gate is equivalent to $\text{T}^2$. By applying the matrix proven above twice, one finds that the resultant matrix
$$
\scalebox{1}{$
\begin{bmatrix}
    0.9999999999999991&0.\\
    0.&0.0000029394225792 + 1.0000050578127595\mathbf{i}
\end{bmatrix}
$}
$$
approximates the S gate to $\epsilon = 5.84993 \times 10^{-6}$.

\paragraph{Proof of H}
The H gate is the most difficult to implement, as it contains an off-axis rotation (around the line X = Z). This leads to a much larger circuit approximation. Without the use of any $\ket{0}$ state ancilla qubits, the H gate can be implemented with the circuit:
\begin{figure}[h]
    \centering
    \scalebox{0.9}{
        \begin{quantikz}
            \ket{\psi_{in}} \: & \qw & \ctrl{1} & \gate{R_Y(\pi/2)^2} & \gate{R_Z(\pi)} & \gate{R_Y(\pi/2)^3} & \: \ket{\psi_{out}}\\
            \ket{\psi_{in}} \: & \gate{R_Y(\pi/2)^3} & \gate{R_Z(\pi)} & \gate{R_Y(\pi/2)^3} & \ctrl{-1} & \gate{R_Y(\pi/2)^2} & \: \ket{0}
        \end{quantikz}}
    \caption{Proof of H Circuit, No $\ket{0}$ Ancilla}
    \label{fig:quantum universal set 1 proof of T circuit}
\end{figure}

Although this can be simplified in the ancilla-allowed construction, it is easy to prove (by means of adding a CNOT gate to the ancilla, controlled on the input) that this method also works allowing for ancilla qubits.

The $R_Z(\pi)$ gate must be approximated with greater prescision to keep the final $\epsilon$ under $10^{-5}$: $R_Z(2\pi \Phi)^{416020}$. Again excluding the ancilla bit, the circuit matrix

$$
\scalebox{1}{$
\begin{bmatrix}
    0.7071067812048173-0.00000119421\mathbf{i}&0.7071067812048173-0.00000119421\mathbf{i}\\
    0.7071067812048173-0.00000119421\mathbf{i}&0.7071067812048173-0.00000119421\mathbf{i}
\end{bmatrix}
$}
$$
is able to approximate the H gate to $\epsilon = 2.38842 \times 10^{-6}$.


\end{document}
